% !TeX spellcheck = es
\documentclass{article}
\usepackage[utf8]{inputenc}
\usepackage{graphicx}
\usepackage{hyperref}
\usepackage{amsmath}
\usepackage{tikz}
\usepackage{float}
\usepackage[simplified]{pgf-umlcd}
\usepackage{subfiles}
\usepackage{listings}
\usepackage[lighttt]{lmodern}
\usepackage{color}
\begin{document}
\section{BiblioCAT}
\subsection{Introducción}
Las libros nos permiten descubrir múndos de fantasia así como entender mejor el mundo en el que vivimos. Nos permiten aprender todo tipo de cosas, desde "Teo va a la escuela" a "Compilers: Principles, Techniques, and Tools".\\\\
%
Las librerias de Cataluña tienen un sistema de intercanvio de libros, de forma que, quando alguien se dirige a una biblioteca a buscar un libro para aprender a programar Haskell, y el/la bibliotecario/a le diga "I eso qué és?!?!?", haya la posibilidad de pedirle a otra biblioteca que tenga el libro que el/la valiente aspirante/a a aprender Haskell pedia.
%
\subsection{Especificaciones}
\begin{itemize}
\item Lo que les gustaria a las bibliotecas és maximizar el tiempo en que los libros estan generando valor, es decir, el tiempo en que un libro esta en manos de algún usuario. \textbf{Cada libro genera un valor X.} 

\item Se ha de tener en cuenta, que mientras un libro esta siendo transportado este \textbf{NO} genera valor. 

\item Dependiendo del lector, i el libro que vaia a leer (en funcion de si le gusta mas o menos), variarà el \textbf{tiempo T} en que se lee el libro.

\item Si hay varias personas que quieren leer un libro, estas se añadiran a una lista de espera, i esperaran a recibir noticias de la biblioteca quando esté disponible.

\item Una persona, como màximo, podra tener tres libros a la vez. \textbf{OJO: NO tres libros por biblioteca, tres libros en TOTAL}. 
\end{itemize}
\subsection{Input File}
\begin{verbatim}
% separados por un espacio %
biblioteca = "L", id, { % { tiempo id j | j <- [ 0 .. id_max] }  % }
id = % numero unico por tipo %
tiempo id j = % tiempo de tranporte de un libro desde la
              - libreria id hasta j
              %
libro = "B", id, id biblioteca, valor
valor = % numero %
lector = "R", id, id biblioteca, { id libro, tiempo lectura }
tiempo lectura = % numero de tiempo que se
                 - tarda en leer el libro id libro
                 %
\end{verbatim}

Por ejemplo:
\begin{verbatim}
L 0 0 2 3
L 1 3 0 4
L 2 3 4 0
B 0 0 5
B 1 0 3
B 2 1 10
B 3 2 2
B 4 2 8
R 0 0 0 10 1 2
R 1 0 1 1 2 10 3 3
R 2 1 0 1 1 1 2 1 3 1 4 1
R 3 2 2 5 4 5
\end{verbatim}
%
En este ejemplo, hay tres bibliotecas (L), donde el coste de llevar un libro
de la biblioteca ``0'' hasta la ``1'' es de 2 i el coste de llevar
un libro de la biblioteca ``1'' hasta la ``0'' es de 3. Además, el coste de
llevar un libro de una biblioteca a si misma siempre es de 0.\\\\
%
En el caso de los libros (B),  el primer argumento es el id del libro, 
el segundo es el id de la biblioteca donde se encuentra
en el estado inicial, y el tercero es el valor que produce el libro en
ser leido por un lector.\\\\
%
Finalmente, se puede apreciar que el lector con id ``0'', que esta afiliado
a la biblioteca ``0'', quiere leer los libros ``0'' y ``1'', que tardará en
hacerlo ``10'' y ``2'' respectivamente.

\subsection{Output File}


\subsection{Normas}
\includegraphics[width=\linewidth]{meme.jpg}

\subsection{Entrega i Evaluación}
Para hacer vàlida vuestra participación nos tendreis que entregar el output de vuestro programa con el formato correcto descrito en el apartado anterior. Para comprovar la score de vuestra solución, podeis descargaros y ejecutar el programa que utilizaremos para determinar vuestra score: [LINK A GITHUB]. 
\end{document}